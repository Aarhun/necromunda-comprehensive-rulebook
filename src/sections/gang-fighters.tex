% Comment this out if you're using the article class.
\onecolumn
\chapter{Gangs Fighters and their weaponry}
In Necromunda, each player controls a ‘gang’, which is made up of a number of models. Each of these models is
referred to as a ‘fighter’ within the rules. Each fighter may have their own rank within the gang or without – Leader,
Ganger, Underhive Scum, Brute and so forth – but the term ‘fighter’ covers them all within the rules.

\section{Characteristics profiles}
Each fighter has a characteristics profile, which describes their capabilities in battle. For example, here is the profile for
a House Orlock Ganger.

\begin{monsterboxnobg}

  \stats[
  M=5,
  WS=4,
  BS=4,
  S=3,
  T=3,
  W=1,
  I=4,
  A=1,
  LD=6,
  Cl=7,
  Will=7,
  Int=7
  ]
\end{monsterboxnobg}

%change layout
\begin{multicols}{2}

\subsection{Move (M)}
This is the distance, in inches, the fighter can move when
making a standard Move action.

\subsection{Weapon Skill (WS)}
This shows the fighter’s proficiency with Melee weapons
and weapons with the Sidearm trait when used in close
combat.

\subsection{Ballistic Skill (BS)}
This shows the fighter’s proficiency with ranged weapons.

\subsection{Strength (S)}
How strong the fighter is. The higher a fighter’s Strength,
the more likely they are to inflict damage on an opponent
in close combat, for example.

\subsection{Toughness (T)}
How tough the fighter is. The higher a fighter’s
Toughness, the less likely they are to be wounded by an
attack.

\subsection{Wounds (W)}
A fighter’s Wounds characteristic is a measure of how
much punishment they can take before succumbing to
their injuries.

\subsection{Initiative(I)}
Initiative is a measure of a fighter’s dexterity and reflexes.

\subsection{Attacks (A)}
This is a measure of a fighter’s speed and ability in melee.
When a fighter is Engaged in close combat, their Attacks
characteristic determines how many dice are rolled when
they attack their enemies.

\subsection{Leadership (LD)}
This is a measure of fighter’s ability to issue or follow
commands in the heat of battle.

\subsection{Cool (Cl)}
A fighter’s Cool represents their capacity for keeping calm
under fire.

\subsection{Willpower (Will)}
Willpower is a measure of fighter’s mental fortitude and
resilience.

\subsection{Intelligence (Int)}
This represents a fighter’s mental acuity and ability to
apply knowledge.

\subsection{Modifying characteristics}
Sometimes, the rules will modify a characteristic. If the
characteristic is given a simple number, the modifier is
applied as written – for example, if a fighter with Strength
3 is given a +1 Strength modifier, their Strength counts as
4.
If the characteristic is given as a target number (for
example, a characteristic of 4+ means a dice roll of 4 or
higher would be a success) the modifier is effectively
applied to the dice roll. For example, if a fighter with
Initiative 4+ is given a +1 Initiative modifier, the
characteristic would be 3+ because a roll of 3 with a +1
modifier applied becomes a roll of 4.


\subsection{Characteristics Checks}
Players will often be called on to make a characteristic
check for a fighter – for example, a Ballistic Skill check is
made when a fighter attacks with a ranged weapon.
Characteristics checks are made as follows:
\begin{itemize}
  \item For Weapon Skill, Ballistic Skill and Initiative, roll a
  D6. If the result is equal to or higher than the
  characteristic, the check is passed.
  \item For Leadership, Cool, Willpower and Intelligence,
  roll 2D6. If the result is equal to or higher than the
  characteristic, the check is passed.
  \item For Strength or Toughness, roll a D6. If the result is
  equal to or lower than the characteristic, the check
  is passed.
\end{itemize}

\end{multicols}

\section{Models and Fighter cards}
Each player’s gang is made up of a number of fighters, each of which is represented by a model on the tabletop and a
Fighter card filled in with their characteristics, equipment and other useful reference information. Blank fighter
cards can be found in the Necromunda: Underhive boxed set and are available separately. Blank Fighter cards with the
logo of each House can be found in the various Tactics cards packs, perfect for Leaders and Champions.



\header{Each Fighter card is split into several areas:}
\begin{dndtable}
  1. The fighter’s name. If they are a Leader or Champion, it will also be shown here. \\
  2. The fighter’s value, in credits. This is only used in the advanced rules. \\
  3. The fighter’s characteristics. The last four (Ld, Cl, Wil, Int) are shaded as a reminder that checks against them are
  made on 2D6 (see page 4). \\
  4. The weapons the fighter is carrying. \\
  5. Any skills the fighter may have. \\
  6. Any equipment (including armour) carried by the fighter.\\
\end{dndtable}

\includegraphics[width=\textwidth]{example-image-a}

\begin{paperbox}{Designer’s Note: The Golden Rule}
  Necromunda is a game with lots of moving parts, and it’s inevitable that rules might sometimes come into conflict.
  When it’s not clear how to proceed, both players should discuss what they think is the most sensible solution – and if
  an agreement cannot be reached, roll off to decide. The most important thing is to not let debates get in the way of
  a fun game !
\end{paperbox}




















\section{Weapon profiles}
tata

\begin{commentbox}{This Is a Comment Box!}
  A \lstinline!commentbox! is a box for minimal highlighting of text. It lacks the ornamentation of \lstinline!paperbox!, but it can handle being broken over a column.
\end{commentbox}

\subtitlesection{Weapon, +1, +2, or +3}
{Weapon (any), uncommon (+1), rare (+2), or very rare (+3)}

% For more columns, you can say \begin{dndtable}[your options here].
% For instance, if you wanted three columns, you could say
% \begin{dndtable}[XXX]. The usual host of tabular parameters are
% available as well.
\header{Nice table}
\begin{dndtable}
   	\textbf{Table head}  & \textbf{Table head} \\
   	Some value  & Some value \\
   	Some value  & Some value \\
   	Some value  & Some value
\end{dndtable}

\section{Spells}

\begin{spell}
	{Beautiful Typesetting}
	{4th-level illusion}
	{1 action}
	{5 feet}
	{S, M (ink and parchment, which the spell consumes)}
	{Until dispelled}
	You are able to transform a written message of any length into a beautiful scroll. All creatures within range that can see the scroll must make a wisdom saving throw or be charmed by you until the spell ends.

	While the creature is charmed by you, they cannot take their eyes off the scroll and cannot willingly move away from the scroll. Also, the targets can make a wisdom saving throw at the end of each of their turns. On a success, they are no longer charmed.
\end{spell}

\lipsum[2]

\section{Colors}

This package provides several global color variables to style \lstinline!commentbox!, \lstinline!quotebox!, \lstinline!paperbox!, and \lstinline!dndtable! environments.

\begin{dndtable}[lX]
  \textbf{Color}         & \textbf{Description} \\
  \lstinline!commentboxcolor! & Controls \lstinline!commentbox! background. \\
  \lstinline!paperboxcolor!   & Controls \lstinline!paperbox! background. \\
  \lstinline!quoteboxcolor!   & Controls \lstinline!quotebox! background. \\
  \lstinline!tablecolor!      & Controls background of even \lstinline!dndtable! rows. \\
\end{dndtable}

See Table~\ref{tab:colors} for a list of accent colors that match the core books.

\begin{table*}
  \begin{dndtable}[XX]
    \textbf{Color}                            & \textbf{Description} \\
    \lstinline!PhbLightGreen!                      & Light green used in PHB Part 1 \\
    \lstinline!PhbLightCyan!                       & Light cyan used in PHB Part 2 \\
    \lstinline!PhbMauve!                           & Pale purple used in PHB Part 3 \\
    \lstinline!PhbTan!                             & Light brown used in PHB appendix \\
    \lstinline!DmgLavender!                        & Pale purple used in DMG Part 1 \\
    \lstinline!DmgCoral!                           & Orange-pink used in DMG Part 2 \\
    \lstinline!DmgSlateGray! (\lstinline!DmgSlateGrey!) & Blue-gray used in PHB Part 3 \\
    \lstinline!DmgLilac!                           & Purple-gray used in DMG appendix \\
  \end{dndtable}
  \caption{Colors supported by this package}%
  \label{tab:colors}
\end{table*}

\begin{itemize}
  \item Use \lstinline!\setthemecolor[<color>]! to set \lstinline!themecolor!, \lstinline!commentcolor!, \lstinline!paperboxcolor!, and \lstinline!tablecolor! to a specific color.
  \item Calling \lstinline!\setthemecolor! without an argument sets those colors to the current \lstinline!themecolor!.
  \item \lstinline!commentbox!, \lstinline!dndtable!, \lstinline!paperbox!, and \lstinline!quoteboxcolor! also accept an optional color argument to set the color for a single instance.
\end{itemize}

\subsection{Examples}

\subsubsection{Using \lstinline!themecolor!}

\begin{lstlisting}
\setthemecolor[PhbMauve]

\begin{paperbox}{Example}
  \lipsum[2]
\end{paperbox}

\setthemecolor[PhbLightCyan]

\header{Example}
\begin{dndtable}[cX]
  \textbf{d8} & \textbf{Item} \\
  1           & Small wooden button \\
  2           & Red feather \\
  3           & Human tooth \\
  4           & Vial of green liquid \\
  6           & Tasty biscuit \\
  7           & Broken axe handle \\
  8           & Tarnished silver locket \\
\end{dndtable}
\end{lstlisting}

\begingroup
\setthemecolor[PhbMauve]

\begin{paperbox}{Example}
  \lipsum[2]
\end{paperbox}

\setthemecolor[PhbLightCyan]

\header{Example}
\begin{dndtable}[cX]
  \textbf{d8} & \textbf{Item} \\
  1           & Small wooden button \\
  2           & Red feather \\
  3           & Human tooth \\
  4           & Vial of green liquid \\
  6           & Tasty biscuit \\
  7           & Broken axe handle \\
  8           & Tarnished silver locket \\
\end{dndtable}
\endgroup

\subsubsection{Using element color arguments}

\begin{lstlisting}
\begin{dndtable}[cX][DmgCoral]
  \textbf{d8} & \textbf{Item} \\
  1           & Small wooden button \\
  2           & Red feather \\
  3           & Human tooth \\
  4           & Vial of green liquid \\
  6           & Tasty biscuit \\
  7           & Broken axe handle \\
  8           & Tarnished silver locket \\
\end{dndtable}
\end{lstlisting}

\begin{dndtable}[cX][DmgCoral]
  \textbf{d8} & \textbf{Item} \\
  1           & Small wooden button \\
  2           & Red feather \\
  3           & Human tooth \\
  4           & Vial of green liquid \\
  6           & Tasty biscuit \\
  7           & Broken axe handle \\
  8           & Tarnished silver locket \\
\end{dndtable}

% End document
\end{document}
