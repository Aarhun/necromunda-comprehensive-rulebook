% Comment this out if you're using the article class.
\onecolumn
\chapter{Gangs Fighters and their weaponry}
In Necromunda, each player controls a ‘gang’, which is made up of a number of models.
Each of these models is referred to as a ‘fighter’ within the rules.
Each fighter may have their own rank within the gang or without – Leader, Ganger, Underhive Scum, Brute and so forth – but the term ‘fighter’ covers them all within the rules.

\section{Characteristics profiles}
Each fighter has a characteristics profile, which describes their capabilities in battle.
For example, here is the profile for a House Orlock Ganger.

% stats table for Orlock Ganger
  %\begin{monsterboxnobg}[before skip=12pt,]
  \begin{monsterboxnobg}

    \stats[
    M=5,
    WS=4,
    BS=4,
    S=3,
    T=3,
    W=1,
    I=4,
    A=1,
    LD=6,
    Cl=7,
    Will=7,
    Int=7
    ]
  \end{monsterboxnobg}
 %

%change layout
\begin{multicols}{2}

\subsection{Move (M)}
This is the distance, in inches, the fighter can move when making a standard Move action.

\subsection{Weapon Skill (WS)}
This shows the fighter’s proficiency with Melee weapons and weapons with the Sidearm trait when used in close combat.

\subsection{Ballistic Skill (BS)}
This shows the fighter’s proficiency with ranged weapons.

\subsection{Strength (S)}
How strong the fighter is.
The higher a fighter’s Strength, the more likely they are to inflict damage on an opponent in close combat, for example.

\subsection{Toughness (T)}
How tough the fighter is.
The higher a fighter’s Toughness, the less likely they are to be wounded by an attack.

\subsection{Wounds (W)}
A fighter’s Wounds characteristic is a measure of how much punishment they can take before succumbing to their injuries.

\subsection{Initiative(I)}
Initiative is a measure of a fighter’s dexterity and reflexes.

\subsection{Attacks (A)}
This is a measure of a fighter’s speed and ability in melee.
When a fighter is Engaged in close combat, their Attacks characteristic determines how many dice are rolled when they attack their enemies.

\subsection{Leadership (LD)}
This is a measure of fighter’s ability to issue or follow commands in the heat of battle.

\subsection{Cool (Cl)}
A fighter’s Cool represents their capacity for keeping calm under fire.

\subsection{Willpower (Will)}
Willpower is a measure of fighter’s mental fortitude and resilience.

\subsection{Intelligence (Int)}
This represents a fighter’s mental acuity and ability to apply knowledge.

\subsection{Modifying characteristics}
Sometimes, the rules will modify a characteristic.
If the characteristic is given a simple number, the modifier is applied as written – for example, if a fighter with Strength 3 is given a +1 Strength modifier, their Strength counts as 4.
If the characteristic is given as a target number (for example, a characteristic of 4+ means a dice roll of 4 or
higher would be a success) the modifier is effectively applied to the dice roll. For example, if a fighter with Initiative 4+ is given a +1 Initiative modifier, the characteristic would be 3+ because a roll of 3 with a +1 modifier applied becomes a roll of 4.


\subsection{Characteristics Checks}
Players will often be called on to make a characteristic
check for a fighter – for example, a Ballistic Skill check is
made when a fighter attacks with a ranged weapon.
Characteristics checks are made as follows:
\begin{itemize}
  \item For Weapon Skill, Ballistic Skill and Initiative, roll a
  D6. If the result is equal to or higher than the
  characteristic, the check is passed.
  \item For Leadership, Cool, Willpower and Intelligence,
  roll 2D6. If the result is equal to or higher than the
  characteristic, the check is passed.
  \item For Strength or Toughness, roll a D6. If the result is
  equal to or lower than the characteristic, the check
  is passed.
\end{itemize}

\end{multicols}

\section{Models and Fighter cards}
Each player’s gang is made up of a number of fighters, each of which is represented by a model on the tabletop and a
Fighter card filled in with their characteristics, equipment and other useful reference information. Blank fighter
cards can be found in the Necromunda: Underhive boxed set and are available separately. Blank Fighter cards with the
logo of each House can be found in the various Tactics cards packs, perfect for Leaders and Champions.



\header{Each Fighter card is split into several areas:}
\begin{dndtable}
  1. The fighter’s name. If they are a Leader or Champion, it will also be shown here. \\
  2. The fighter’s value, in credits. This is only used in the advanced rules. \\
  3. The fighter’s characteristics. The last four (Ld, Cl, Wil, Int) are shaded as a reminder that checks against them are
  made on 2D6 (see page 4). \\
  4. The weapons the fighter is carrying. \\
  5. Any skills the fighter may have. \\
  6. Any equipment (including armour) carried by the fighter.\\
\end{dndtable}

%\includegraphics[width=\textwidth]{example-image-a}

\begin{paperbox}{Designer’s Note: The Golden Rule}
  Necromunda is a game with lots of moving parts, and it’s inevitable that rules might sometimes come into conflict.
  When it’s not clear how to proceed, both players should discuss what they think is the most sensible solution – and if an agreement cannot be reached, roll off to decide.
  The most important thing is to not let debates get in the way of a fun game !
\end{paperbox}




\section{Weapon profiles}
tata

\begin{commentbox}{This Is a Comment Box!}
  A \lstinline!commentbox! is a box for minimal highlighting of text. It lacks the ornamentation of \lstinline!paperbox!, but it can handle being broken over a column.
\end{commentbox}


